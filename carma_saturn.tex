Science case

A. Calibration of high resolution CMB polarization experiments:

Cosmic microwave background (CMB) experiments have revolutionized our understanding of cosmology, allowing high precision measurements of the cosmic parameters that characterize the standard cosmological model. The next frontier of CMB research is focused on exploiting the polarized CMB anisotropy to make a high-precision measurement of gravitational lensing of the CMB by large-scale structure, and search for the elusive gravitational wave background originating from an inflationary epoch.

Experiments aimed at making high-precision measurements of the CMB anisotropy require accurate knowledge of the angular response of the instrument and absolute calibration to a few percent level accuracy or better. For experiments targeting the CMB polarization signal the polarization fraction and polarization orientation must also be measured to high accuracy. These latter measurements necessarily depend on a bright and well-measured polarized source. It is common for CMB experiments, both ground- and space-based, to use bright sources, such as planets or fixed celestial sources for absolute calibration and to measure beam profiles. The calibration of the polarization amplitude and orientation presents more of a challenge as non-variable and compact celestial calibrators that are sufficiently polarized are not common at millimeter wavelengths.
 
Previous work on calibration of the CMB polarization angle has focussed on fixed nearby celestial sources, such as HII regions or supernovae remnants e.g. Tau A. However, these sources in addition to potentially being variable are spatially extended sources for CMB experiments with arcminute resolution, such as ACTPol (*ACTPol-Niemack). This means that for high-resolution CMB experiments these sources fill the beam and more detailed modeling of the source polarization is required to accurately measure the position angle, making them difficult to utilize as polarization calibrators. Another possibility is to calibrate the polarization amplitude and position angle using extragalactic point sources, however, extrapolation from  polarized observations of these sources at centimeter wavelengths, is subject to a frequency dependent and time varying polarization amplitude and position angle, making them far less reliable as calibrators.

Another bright polarized source at millimeter wavelengths, that has not been previously considered by CMB polarization experiments, is Saturn. Saturn is bright enough to be detected as a microwave source (*WMAP-Weiland), and even though a more detailed model of the disk+ring system is required for it to be used as a high-precision temperature calibrator, it has been used for high-precision beam profile measurements e.g. for the ACT experiment (*Hincks et al.). Its relatively high degree of polarization that arises from scattering of Saturnian light -- Grossman et al, 1989 indicate an average polarization of 5% -- makes it suitable as a polarization calibrator, In addition, an overall net polarization is expected from Saturn's rings for experiments that do not resolve the rings (Fig. 3 of Grossman et al. 1989). Calibration of the polarization amplitude may require a more detailed modeling of Saturn's disk+ring system, though other methods can be used to calibrate the CMB polarization amplitude, such as calibrating the polarization amplitude off the measured cross temperature-polarization anisotropy from WMAP. At a minimum, Saturn's polarization at millimeter wavelengths will provide a high precision calibrator of the polarization position angle of CMB experiments. In this proposal, we seek to make a high-precision measurement of Saturn's polarized emission at millimeter wavelengths with CARMA.

The proposed observations on CARMA will also allow interesting planetary science by probing the composition and distribution of particles in Saturn's rings, as we now describe.

B. Constraining the scattering properties of particles in Saturn's rings:

Observations of a polarized signal from the rings at 1.3 mm would complement existing observations at centimeter wavelengths  (Grossman et al. 1989, van der Tak et al. 1999) and provide new insights into the ring's particle properties and structure. Models of the rings' microwave spectra indicate that at wavelengths longward of 1 centimeter, the rings' brightness is dominated by scattered radiation from Saturn (Fig. 7 of Dunn et al. 2005). Indeed, variations in the orientation and intensity of the polarized centimeter-wave radiation along the rings are consistent with such a model. By contrast, at shorter wavelengths, thermal emission from the rings themselves should also make a large contribution to the rings' total brightness (Fig. 7 of Dunn et al. 2005). Polarization data provides a way to measure the relative strength of these two components, which in turn can help constrain the ring particles' size distribution and composition (Dunn et al. 2002).

If the rings were composed exclusively of isolated, spherical particles, then the thermal emission should not have a strong net polarization. In this case, the polarized component in the rings' millimeter-wave emission would come entirely from the scattered Saturn-shine, and determining the relative contributions of the two components would be relatively straightforward. However, the real situation will likely be more complicated since particles in Saturn's dense rings are not isolated. Indeed, in many parts of the A and B rings the particles are organized into elongated aggregates known as self-gravity wakes (Colwell et al. 2009). Such structures could give rise to a polarized component in the ring's thermal emission (Edgington et al. 2010), but since self-gravity wakes are canted relative to the radial direction, the polarized signal they generate should have different orientation from the polarization generated by scattered Saturn-shine. Hence isolating the contributions from the two components would still be possible, and constraints on the rings' light-scattering properties could still be derived.



Technical Justification.
Somewhat surprisingly, there are no good polarization data on Saturn
at mm wavelengths.  Since {\br{polarization is unavailable at 3mm, we
    request 1mm time}}.  The extent of the Saturnian system is
somewhat larger than the 10m primary beams at 1mm, so we must
{\bf{image}} the system in order to model the polarized signal.  From
Dunn et al. (2005) the expected surface brightness of Saturn's rings
due to scattered light at 1mm is about 2K.  Since thermal emission
should be unpolarized, we do not expect the thermal component of
Saturn's rings' emission to contribute to the polarized signal.  At
2cm, Dunn et al. estimate the polarization fraction of the reflection
component to be 10\%, so the {\bf{brightness temperature of the rings
    in polarization at 1mm should be $\sim$200 mK}}.  The apparent
thickness of the ring system at Saturn's current inclination of 17
degrees ranges from $\sim 2-5$ arcseconds, so we request {\bf{E-array continuum
    time, though D-array would likely work as well}.  Saturn is
currently at a declination of -8 degrees.  The CARMA sensitivity
calculator gives the brightness sensitivity to be 0.72 mK (including
the effects of 8\% shadowing) after a single track, for a
single-pointing SNR of 300/beam.  To get a good model, we wish to
cover the rings and the disk of Saturn, which are comparable to the
10m primary beams at 1mm.  We propose to carry out a {\bf{9-point
    mosaic}} - a standard 7 point hexagonal mosaic aligned with the
rings, with an additional pointing on either side, oriented along the
axis of the rings.  This would give us 5 pointings along the rings -
if separated by the FWHM of the 10m dishes, then the mosaic extent
compares well with the apparent diameter of Saturn's A-ring (the
outermost of the primary rings) of 44'' at opposition, with an SNR/beam in
polarization of 100.  Integrating over the disk area easily gets us to
our desired net polarization sensitivity of 0.2 degrees, or and SNR of
300.  We therefore request {\bf{one track}}.  We note that even were
we to need greater sensitivity, we would wish to make sure we're not
limited by systematics before requesting further tracks.

The proposers have extensive experience working with interferometric
data, including in polarization.  However, it is invariably the case
that when  one wishes to push to percent-level accuracies, a deep
understanding of the instrument is required.  We therefore request
consultation with CARMA staff experienced at polarization
measurements.  While we do not request collaboration, were CARMA staff
interested in doing so, we would welcome them.  


References:

Grossman et al. 1989 Science 245: 1211-1215
van der Tak et al. 1999 Icarus 142: 125-147
Dunn et al. 2005 AJ 129: 1109-1116
Dunn et al. 2002 Icarus 160: 132-160
Colwell et al. 2009 in Saturn From Cassini Huygens by Dougherty et al.
Edgington et al. 2010 BAAS 42: 1008 (DPS meeting abstract)
