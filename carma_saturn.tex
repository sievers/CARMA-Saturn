Somewhat surprisingly, there are no good polarization data on Saturn
at mm wavelengths.  Since {\br{polarization is unavailable at 3mm, we
    request 1mm time}}.  The extent of the Saturnian system is
somewhat larger than the 10m primary beams at 1mm, so we must
{\bf{image}} the system in order to model the polarized signal.  From
Dunn et al. (2005) the expected surface brightness of Saturn's rings
due to scattered light at 1mm is about 2K.  Since thermal emission
should be unpolarized, we do not expect the thermal component of
Saturn's rings' emission to contribute to the polarized signal.  At
2cm, Dunn et al. estimate the polarization fraction of the reflection
component to be 10\%, so the {\bf{brightness temperature of the rings
    in polarization at 1mm should be $\sim$200 mK}}.  The apparent
thickness of the ring system at Saturn's current inclination of 17
degrees ranges from $\sim 2-5$ arcseconds, so we request {\bf{E-array continuum
    time, though D-array would likely work as well}.  Saturn is
currently at a declination of -8 degrees.  The CARMA sensitivity
calculator gives the brightness sensitivity to be 0.72 mK (including
the effects of 8\% shadowing) after a single track, for a
single-pointing SNR of 300/beam.  To get a good model, we wish to
cover the rings and the disk of Saturn, which are comparable to the
10m primary beams at 1mm.  We propose to carry out a {\bf{9-point
    mosaic}} - a standard 7 point hexagonal mosaic aligned with the
rings, with an additional pointing on either side, oriented along the
axis of the rings.  This would give us 5 pointings along the rings -
if separated by the FWHM of the 10m dishes, then the mosaic extent
compares well with the apparent diameter of Saturn's A-ring (the
outermost of the primary rings) of 44'' at opposition, with an SNR/beam in
polarization of 100.  Integrating over the disk area easily gets us to
our desired net polarization sensitivity of 0.2 degrees, or and SNR of
300.  We therefore request {\bf{one track}}.  We note that even were
we to need greater sensitivity, we would wish to make sure we're not
limited by systematics before requesting further tracks.

The proposers have extensive experience working with interferometric
data, including in polarization.  However, it is invariably the case
that when  one wishes to push to percent-level accuracies, a deep
understanding of the instrument is required.  We therefore request
consultation with CARMA staff experienced at polarization
measurements.  While we do not request collaboration, were CARMA staff
interested in doing so, we would welcome them.  

